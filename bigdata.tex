%% bigdata.tex
%% Jeremy Singer
%% 30 July 2015

\documentclass[conference,10pt]{IEEEtran}


\usepackage{hyperref}

\begin{document}

\title{Economic Incentives for free market cloud computing}


% author names and affiliations
% use a multiple column layout for up to three different
% affiliations
\author{\IEEEauthorblockN{Jeremy Singer \and Abyd Adhami}
\IEEEauthorblockA{School of Computing Science\\
University of Glasgow\\
UK\\
Email: jeremy.singer@glasgow.ac.uk}
}

\maketitle


\begin{abstract}
Although cloud computing has made great inroads in terms of consumer adoption, we argue that the paradigm still has a long way to go to reach an economically efficient free market state.
We highlight three key problems with the current cloud computing approach. These are (1) coarse granularity of service provision, (2) inadequate support for small providers and prosumers, and (3) vendor lockin via APIs.
In this position paper, we outline interesting lines of work that may lead to some solutions to these problems. The solutions are all based on our experiences with applying economic theory to computer systems.
\end{abstract}

%%%%%%%%%%%%

\section{Introduction}

Utility computing is now a well-established concept. Brief history of cloud.
Current stats. Why is cloud computing so popular? NIST definition. Talk about capex being replaced by opex.

However - not yet fully realised the power of this computing model - service provision/delivery. In this paper, we will outline three problems with cloud computing - preventing its full adoption - realising its potential - in a sustainable society.

Argue that with increasing data (big data) on cloud provision - computer systems in general - makes the problems more acute.??? Harness big data - to make a free market for all. 

Generally focus on infrastructure-as-a-service (IaaS) since this is where I have most experience. Same arguments will apply to other levels of cloud service provision.

%%%%%%%%%%

\section{Paradigm Problems}

This section considers three issues with current cloud computing service provision.
These issues prevent a free market economy from operating efficiently.

%%%%%%

\subsection{Coarse Granularity of Services}

In terms of IaaS provision, most providers offer predefined, quantized units of computing resource. For example, the lowest specification of an Amazon Web Services instance is currently \emph{t2.micro} which features a single CPU and 1 GB RAM. Although the CPU bursting concept allows some rewards for unused CPU cycles, the rewards can only be claimed in terms of subsequent processing, rather than refunds or cheaper pricing.  

Another quantized unit is the minimum time period for which instances can be commissioned. AWS customers must rent machines by the hour. Google Compute Engine customers are charged for at least 10 minutes.
Although there are some more flexible services, e.g.\ the asynchronous event-based AWS Lambda platform, such innovations are not as flexible and are not yet commonplace.

This quantization of service is inefficient, in terms of market economics. Very often, consumers will pay for more provision than they actually require. Another disadvantage is that any kind of statistical analysis, modeling or forecasting for computing requirements would be more straightforward with continuous parameters. 

Finally, the different instance types hinder fair comparison. Often, it is not possible to make direct comparisons between different IaaS providers since they are offering products that are not precisely equivalent.

%%%%%%%

\subsection{Exclusion of Small Providers}

According to a recent industrial survey, the \emph{big four} providers of compute infrastructure, (Amazon, Microsoft, Google, IBM) control over 50\% of the cloud market \footnote{\url{http://www.eweek.com/cloud/big-four-iaas-providers-now-own-half-the-market.html}}.

In terms of physical requirements, there is a great deal of overhead in building and maintaining a datacenter. Significant costs include electricity. Unexpected costs include compliance with ISO certification standards for cloud security.

These significant barriers to entry prevent small providers from participating in the cloud service provision market. 

Examples - Raspberry Pi cloud - others - micro datacenters.


Also - prosumers - individuals contributing cycles - either for research (BOINC) or charitable ?? purposes. Grid computing. Also applicable to consumer clouds. Highlight parallels with electricity market - solar panels feeding back into grid. Similar with individuals with computers?

No way to handle this / harness this at the moment?
Require some kind of cloud brokerage service.
Give some examples- e.g. cloudaroo, rightscale.

%%%%%%%%

\subsection{Incompatibility of Utilities}

Each of the major cloud service providers has its own custom APIs and functionality for major features. Examples - data storage. Key/value stores - block storage, etc. Similarly, for emailing - etc. 
Why? No cross-industry standards? 
Basic vendor lockin. Ideally - want some kind of facade on top - some of these available but none widely adopted. Need something like POSIX compliance for Operating Systems - need a similar idea for cloud providers.
So easy to migrate services dynamically, from one provider's cloud to another.

might argue that we stick with lowest common denominator - e.g. an x86 Linux server and build all services on top - e..g database etc. BUT having to administer all these services oneself is a big overhead - lose many of the benefits of utility computing in the first place (one of which is minimal setup and configuration).

%%%%%%%%%%%%%%%%

\section{Research Directions}

(abyd's text here)

\section{Related Work}

Look at Agoric systems - summarize this work.
Look at Grid systems - highlight some survey articles here.
Look at economics of cloud - some ideas here too.

\section{Concluding Remarks}

Early stages - cloud computing still developing. Long way to go.
%%%

\end{document}
